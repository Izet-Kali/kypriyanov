\section{Технические требования}

\subsection{Требования назначения}

Требования к электрическим параметрам изделия представлены в таблице \ref{tab:teh-treb-ez}.




\begin{longtable}{|p{4cm}|p{2cm}|p{2cm}|p{2cm}|p{2cm}|p{2cm}|}
\caption{Требования к электрическим параметрам изделия}\label{tab:teh-treb-ez}
\hline
\multirow{2}{4cm}{\textbf{Параметр}}                   & \multirow{2}{2cm}{\textbf{Обознач-ение}}    & \multicolumn{3}{c|}{\textbf{Норма}} & \multirow{2}{2cm}{\multicolumn{1}{c}{\textbf{Примечание}}} \\ \cline{3-5}
& & \textbf{Не менее} & \textbf{Номинал} & \textbf{Не более} & \\ \hline 
\endfirsthead    % Первые ячейки начала таблицы
\caption*{Продолжение таблицы \ref{tab:teh-treb-ez}}%
\hline
\multirow{2}{4cm}{\textbf{Параметр}}                   & \multirow{2}{2cm}{\textbf{Обознач-ение}}    & \multicolumn{3}{c|}{\textbf{Норма}} & \multirow{2}{2cm}{\multicolumn{1}{c}{\textbf{Примечание}}} \\ \cline{3-5}
& & \textbf{Не менее} & \textbf{Номинал} & \textbf{Не более} & \\ \hline 
\endhead %первые ячейки после каждого переноса на новую строку
Диапазон рабочих частот, ГГц & $\Delta f_\text{раб.}$ &  & & & \\ 
-- Нижняя граница & & 8 & & & \\ 
-- Номинал & & & Не нормируется & & \\ 
-- Верхняя граница & & & & 12 & \\ \hline 
Диапазон частот выходного сигнала, полоса частот, МГц & $\Delta f_\text{вых.}$ & 300 & 500 & 700 & \\ \hline
Диапазон регулировки фазы, ° &  & 0 & Не нормируется & 360 & \\
Шаг регулировки фазы, ° &  & 10 & 11 & 12 & \\
Точностью установки фазы, ° &  & 5 & 8 & 20 & \\ \hline
Коэффициент передачи, дБ & $K_\text{пер.}$ & 30 & 32 & 35 & \\ \hline
Коэффициент шума, дБ & $K_\text{ш}$ & Не нормируется & 4 & 10 & \\ \hline
Неравномерность коэффициента передачи, дБ & $\Delta K_\text{пер.}$ & - & - & 3 & \\ \hline
%Мгновенный динамический диапазон в режиме <<узкая полоса>>(при компрессии $K_\text{пер.} = 1$ дБ), дБм & $\text{ДД}_\text{мги}$ & 85 & - & - & (75 - максимальное значение)\\ \hline
Верхняя граница линейности амплитудной характеристики по выходу полосы частот (при компрессии $K_\text{пер.} = 1$ дБ), дБм & $P_\text{лин. вых.}$ & 26 & 28 & 30 &  \\ \hline

\end{longtable}


Протокол управления: SPI; Количество каналов управления: 4; Уровни управляющих сигналов: 1,8 и(или) 3,3; 




\subsection{Требования живучести}

Обеспечить сохранение работоспособности изделия в течении всего срока службы, срок службы изделия определяется документацией.

\subsection{Требования надежности}

Наработка до отказа системы из двух тысяч модулей - 1000 часов.


\subsection{Требования стойкости к климатическим воздействиям}

Настоящее изделие согласно \textit{ГОСТ-15150-69} относится к категории \textit{W5.1}. Категория \textit{W5.1} - это изделия, предназначенные для эксплуатации во всех макроклиматических районах на суше и на море, кроме климатического района с антарктическим холодным климатом (всеклиматическое исполнение). Для эксплуатации в качестве встроенных элементов внутри комплектных изделий, предназначенных для эксплуатации в помещениях (объемах) с повышенной влажностью (например, в неотапливаемых и невентилируемых подземных помещениях, в том числе шахтах, подвалах, в почве, в таких судовых, корабельных и других помещениях, в которых возможно длительное наличие воды или частая конденсация влаги на стенах и потолке, в частности в некоторых трюмах, в некоторых цехах текстильных, гидрометаллургических производств и т.п.), конструкция которых исключает возможность конденсации влаги на встроенных элементах (например, внутри радиоэлектронной аппаратуры).\cite{GOST-1}


Требуемые значения рабочих температур представлены в таблицах \ref{tab:treb-temp-shadow}, \ref{tab:treb-temp-solar}.

\begin{table}[H]
    \centering
    \begin{tabular}{|c|c|c|c|}
    \hline
    \multicolumn{2}{|c|}{\textbf{Рабочая температура}} & \multicolumn{2}{c|}{\textbf{Предельная рабочая температура}}\\ \hline
       \textbf{Верхний предел} & \textbf{Нижний предел} & \textbf{Верхний предел} & \textbf{Нижний предел} \\ \hline
       \multicolumn{1}{|c|}{$+45^\circ C$}& $-40^\circ C$ & $+45^\circ C$ & $-40^\circ C$ \\ \hline
    \end{tabular}
    \caption{Требуемые рабочие температуры в тени}
    \label{tab:treb-temp-shadow}
\end{table}




\begin{table}[H]
    \centering
    \begin{tabular}{|c|c|c|c|}
    \hline
    \multicolumn{2}{|c|}{\textbf{Рабочая температура}} & \multicolumn{2}{c|}{\textbf{Предельная рабочая температура}}\\ \hline
       \textbf{Верхний предел} & \textbf{Нижний предел} & \textbf{Верхний предел} & \textbf{Нижний предел} \\ \hline
    $+60^\circ C$ & $-25^\circ C$ & $+60^\circ C$ & $-25^\circ C$ \\ \hline
    \end{tabular}
    \caption{Требуемые рабочие температуры под воздействием солнечного излучения}
    \label{tab:treb-temp-solar}
\end{table}


Требования к рабочей влажности воздуха представлены на таблице \ref{tab:rab-vlaj-vozd}.

\begin{table}[H]
    \centering
    \begin{tabular}{|l|l|p{7cm}|}
        \hline
         \textbf{Средне годовая} & \textbf{Верхнее значение} & \textbf{Абсолютная влажность среднегодовое значение} \\ \hline
         \multicolumn{1}{|c|}{ 80\% при $27^\circ C$ } & \multicolumn{1}{c|}{98\% при $35^\circ C$} & \multicolumn{1}{c|}{20}\\ \hline 
    \end{tabular}
    \caption{Требования к рабочей влажности воздуха}
    \label{tab:rab-vlaj-vozd}
\end{table}

Требования к атмосферному давлению представлены в таблице \ref{tab:treb-davl}.

\begin{table}[H]
    \centering
    \begin{tabular}{|c|c|}
    \hline
        \multicolumn{2}{|c|}{\textbf{Атмосферное давление, \[кПа\]}}  \\ \hline
        \textbf{Верхний предел} & \textbf{Нижний предел} \\ \hline
        106,7 & 84,0 \\ \hline
    \end{tabular}
    \caption{Требования к давлению воздуха в процессе работы устройства}
    \label{tab:treb-davl}
\end{table}

\subsection{Требования к технологичности}

Технологические ограничения к параметрам печатной платы представлены в таблице \ref{tab:teh-treb-pp}. Допустимые материалы печатной платы:




\begin{itemize}
    \begin{minipage}{.30\textwidth}
        \item FR-4;
        \item RO4003C;
        \item RO4350B;
        \item WL-CT338;
    \end{minipage}
    \begin{minipage}{.30\textwidth}
        \item WL-CT350;
        \item F4BM255;
        \item WL-PP350.
    \end{minipage}
\end{itemize}





\begin{longtable}{|p{5cm}|p{2cm}|p{3cm}|p{4cm}|}
\caption[123]{Ограничения к параметрам печатной платы}\label{tab:teh-treb-pp}
\hline
\textbf{Параметр}                    & \textbf{Толщина металлизации} & \textbf{Нижний предел, мм} & \textbf{Верхний предел, мм}  \\ \hline
\endfirsthead    % Первые ячейки начала таблицы
\caption*{Продолжение таблицы \ref{tab:teh-treb-pp}}%
\hline
\textbf{Параметр}                    & \textbf{Толщина металлизации} & \textbf{Нижний предел, мм} & \textbf{Верхний предел, мм}  \\ \hline
\endhead %первые ячейки после каждого переноса на новую строку
\multirow{4}{5cm}{Ширина проводника} & 18                            & 0,125                      & Не нормируется \\ \cline{2-4}
                                   & 35                            & 0,200                        & Не нормируется \\ \cline{2-4}
                                   & 70                            & 0,300                        & Не нормируется \\ \cline{2-4}
                                   & 105                           & 0,350                       & Не нормируется \\ \hline
\multirow{4}{5cm}{Зазор между проводниками} & 18                     & 0,125                      & Не нормируется \\ \cline{2-4}
                                   & 35                            & 0,200                        & Не нормируется \\ \cline{2-4}
                                   & 70                            & 0,300                        & Не нормируется \\ \cline{2-4}
                                   & 105                           & 0,350                       & Не нормируется \\ \hline
\multirow{4}{5cm}{Ширина проводника спирального типа}
& 18                            & 0,200                      & Не нормируется \\ \cline{2-4}
                                   & 35                            & 0,250                        & Не нормируется \\ \cline{2-4}
                                   & 70                            & 0,300                        & Не нормируется \\ \cline{2-4}
                                   & 105                           & 0,350                       & Не нормируется \\ \hline
\multirow{4}{5cm}{Зазор между проводниками спирального типа} & 18                     & 0,120                      & Не нормируется \\ \cline{2-4}
                                   & 35                            & 0,250                        & Не нормируется \\ \cline{2-4}
                                   & 70                            & 0,300                        & Не нормируется \\ \cline{2-4}
                                   & 105                           & 0,350                       & Не нормируется \\ \hline
\multirow{4}{5cm}{Зазор между полигоном и элементами остальной топологии}& 18                     & 0,200                      & Не нормируется \\ \cline{2-4}
                                   & 35                            & 0,200                        & Не нормируется \\ \cline{2-4}
                                   & 70                            & 0,300                        & Не нормируется \\ \cline{2-4}
                                   & 105                           & 0,350                       & Не нормируется \\ \hline
\multirow{4}{5cm}{Параметры сетчатого полигона} & 18                     & 0,200                      & Не нормируется \\ \cline{2-4} 
                                   & 35                            & 0,200                        & Не нормируется \\ \cline{2-4}
                                   & 70                            & 0,300                        & Не нормируется \\ \cline{2-4}
                                   & 105                           & 0,350                       & Не нормируется \\ \hline
\multirow{4}{5cm}{Зазор между открытыми от маски элементами топологии при финишном покрытии ПОС-63} & 18                     & 0,200                      & Не нормируется \\
& & & \\ \cline{2-4}
                                   & 35                            & 0,250                        & Не нормируется \\ \cline{2-4}
                                   & 70                            & 0,300                        & Не нормируется \\ \cline{2-4}
                                   & 105                           & 0,350                       & Не нормируется \\ \hline
Металлизированное отверстие & --                     & 0,300                      & Не нормируется \\ \hline
Отступ элементов топологии от металлизированного отверстия на внутренних слоях & --                     & 0,250                      & Не нормируется \\ \hline
 Поясок монтажной контактной площадки & --                     & 0,200                      & Не нормируется \\ \hline
 Поясок площадки переходного отверстия & --                     & 0,150                      & Не нормируется \\ \hline
Отношение диаметра минимального металлизированного отверстия к толщине печатной платы & --                     & до 1:7, толщина ПП ≤ 2,5 мм                      & до 1:5, толщина ПП > 2,5 мм \\ \hline
Отношение глубины сверления к диаметру металлизированного отверстия & --                     & Не нормируется                      & Не нормируется \\ \hline
Диаметр межслойного переходного отверстия & -- & 0,300  & Не нормируется  \\ \hline

Минимальный диаметр монтажного отверстия & -- & 0,600  & Не нормируется \\ \hline
Диаметр металлизированного полуотверстия & -- & 0,600  & Не нормируется \\ \hline
Минимальное расстояние между краями двух отверстий & -- & 0,200  & Не нормируется \\ \hline
Неметаллизированное отверстие & -- & 0,500  & Не нормируется \\ \hline
Отступ элементов топологии от неметаллизированного отверстия на всех слоях & -- & 0,200  & Не нормируется \\ \hline
Отступ элементов топологии от фрезеруемых контуров на внешних слоях & -- & 0,250  & Не нормируется \\ \hline
Отступ элементов топологии от фрезеруемых контуров на внутренних слоях & -- & 0,250  & Не нормируется \\ \hline
Отступ элементов топологии от края печатной платы на внешних слоях при скрайбировании & -- & 0,400  & Не нормируется \\ \hline
Отступ элементов топологии от края печатной платы на внутренних слоях при скрайбировании & -- & 0,400  & Не нормируется \\ \hline
Минимальная толщина платы & -- & 0,200  & Не нормируется \\ \hline
Минимальная ширина линии маркировки & -- & 0,150  & Не нормируется \\ \hline
Минимальная высота шрифта маркировки & -- & 1  & Не нормируется \\ \hline
Вскрытие текста в маске по текстолиту шириной не менее & -- & 0,150  & Не нормируется \\ \hline
Вскрытие текста в маске по сплошному металлу шириной не менее & -- & 0,250  & Не нормируется \\ \hline
\end{longtable}

\subsection{Требования к документации}


\begin{enumerate}
    \item Вся документация должна соответствовать действующим стандартам ЕСКД, ЕСТД, ЕСПД;
\end{enumerate}


\subsection{Требования приемки работы}

Работа считается выполненной с момента выполнения последнего пункта данного перечня:

\begin{enumerate}
\item В течении 30 рабочих дней с момента подписания контракта исполнитель обязан согласовать заказчиком комплектность конструкторской, технологической, программной документации;
\subparagraph{Примечание:}
В комплект документации обязательно должны войти следующие документы:
    \begin{enumerate}
        \item Схема Е1;
        \item Схема Э3;
        \item Перечень компонентов к схеме Э3.\footnote{перечень компонентов выполняется в виде отдельного документа}
        \item Чертеж печатной платы;\footnote{выполняется согласно ГОСТ 2.417-91}
        \item Сборочный чертеж устройства;
        \item Спецификация;
    \end{enumerate}
\item В течении 60 рабочих дней с момента выполнения пункта 1 исполнитель обязан согласовать перечень отчетной документации с заказчиком;
\item В течении 10 рабочих дней с момента выполнения пункта 2 исполнитель обязан предоставить отчет о проделанной работе в свободной форме с приложенной отчетной документацией;
\item В течении 30 рабочих дней с момента выполнения пункта 3 исполнитель согласует перечень отчетной технологической документации;
\item В течении 10 рабочих дней с момента выполнения пункта 4 исполнитель предоставляет отчет заказчику о проделанной работе с приложенной технологической документацией;
\item В течении 60 рабочих дней с момента выполнения пункта 5 исполнитель производит пять опытных образцов;
\item В течении 10 рабочих дней с момента выполнения пункта 6 исполнитель оформляет и согласует методику проведения испытаний с заказчиком и требования к протоколу проведения испытаний;
\item В течении 20 рабочих дней с момента выполнения пункта 7 исполнитель проводит испытания опытных образцов;
\item В течении 10 рабочих дней с момента выполнения пункта 8 исполнитель предоставляет отчет о проделанной работе в свободной форме, к отчету прилагаются пять протоколов проведения испытаний и пять опытных образцов;
\item В течении 20 рабочих дней с момента выполнения пункта 9 исполнитель формирует комплект документации на изделие и предоставляет его заказчику.
\end{enumerate}

\paragraph{Примечание} 
В случае не выполнения пунктов из перечня в поставленный срок на исполнителя будут накладываться санкции, в соответствии с нормативными документами, .установленными заказчиком на момент подписания контракта.

