\appendix
\renewcommand{\thesection}{Приложение \Asbuk{section}}
\newcommand{\appsec}{\newpage\section{}\setcounter{page}{1}\setcounter{equation}{0}}
% Сбросил номер страницы, номер формулы

\appsec
\section*{Расчет гамма-процентной наработки на отказ}

Наработка на отказ рассчитывается по формуле \ref{app:equationone}.

\begin{equation}\label{app:equationone}
    T_0 = \frac{1}{\Lambda_\text{РЭУ}};
\end{equation}
Где $T_0$ - наработка на отказ, $\Lambda_\text{РЭУ}$ - суммарная интенсивность отказов РЭУ с учетом электрического режима и условий эксплуатации.  

Гамма-процентная наработка до отказа рассчитывается по формуле \ref{app:equationtwo}.

\begin{equation}
\label{app:equationtwo}
T_\Gamma = - \frac{ln\left(\frac{\gamma}{100}\right)}{\Lambda_\text{РЭУ}} = -T_{0}\times ln\left(\frac{\gamma}{100}\right);
\end{equation}Где $\gamma = 90$ - вероятность безотказной работы.

Исходя из условий: наработка на отказ комплекса из 2000 модулей: 1000 часов. Наработка на отказ одного модуля будет  $T_1 \approx 0,5$.


Гамма-процентная наработка до отказа одного модуля:

\begin{equation*}
	T_{\Gamma1} = -0,5 \times ln\left(\frac{90}{100}\right) = -0,52 ;
\end{equation*}

\appsec


\begin{figure}[H]
    \centering
    \fbox{
\begin{circuitikz} 
\draw (0,0) node[mixer, box only, anchor=east, fill=cyan](m){}
(m.west) node[inputarrow]{} to[short, -o] ++(m.west) node[right=-30]{$I_{(t)}$} ++(-0.8,0);
\draw (m.south) node[inputarrow,rotate=90]{} --
++(0,-0.7) node[oscillator, box, anchor=north, fill=cyan]{} -- ++ (0,-1.6) node[inputarrow,rotate=270]{} node[circulator, box, anchor=north, fill=cyan](fase){};
\draw (fase.e) node[right] {$90^\circ$};
\draw (0,-5) node[mixer, box only, anchor=east, fill=cyan](mm){}
(mm.west) node[inputarrow]{} to[short, -o] (-2,-5)
(mm.west) node[right=-60]{$Q_{(t)}$};
\draw (fase.s) -- (mm.n) node[inputarrow,rotate=270]{};
\draw (3,-2.5) node[adder, box only, anchor=east, fill=cyan](sum){}
(m.e);
\draw (m.e)  -|  (sum.n) node[inputarrow,rotate=270]{};
\draw (mm.e)  -|  (sum.s) node[inputarrow,rotate=90]{};
\draw (sum.e)  -- (4,-2.5)
 (4,-2.5) node[inputarrow]{};
\end{circuitikz}
    }
    \caption{Квадратурный модулятор}
%    \label{fig:my_label}
\end{figure}



%\newpage
%\pagestyle{empty}
%\setcounter{page}{1}% С какой страницы начать отсчёт
%\section*{Приложение Б}
