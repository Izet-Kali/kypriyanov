\newpage

\begin{thebibliography}{3}
\addcontentsline{toc}{section}{Список использованных источников}

%





\bibitem{GOST-1}
ГОСТ-15150-69. Машины, приборы и другие технические изделия
\bibitem{GOST-2}
ГОСТ-РВ15.203-2000. Система разработки и постановки продукции на производство. ПРОДУКЦИЯ ПРОИЗВОДСТВЕННО-ТЕХНИЧЕСКОГО НАЗНАЧЕНИЯ. Порядок разработки и постановки продукции на производство
\bibitem{GOST-3}
ГОСТ 23221-78. Модули СВЧ, блоки СВЧ. Термины, определения и буквенные обозначения.
\bibitem{GOST-4}
ГОСТ Р 55787—2013. УСТРОЙСТВА ДЛЯ РАДИОСВЯЗИ, РАДИОВЕЩАНИЯ И ТЕЛЕВИДЕНИЯ АНТЕННО-ФИДЕРНЫЕ.
%\bibitem{GOST-?}
%ГОСТ Р 55744-2013. Платы печатные. Методы испытаний физических параметров.
\bibitem{GOST-5}
ГОСТ 24375-80. РАДИОСВЯЗЬ. Термины и определения.
\bibitem{GOST-6}
ГОСТ 2.417-91. ПЛАТЫ ПЕЧАТНЫЕ. Правила выполнения чертежей.
\bibitem{statiya-1}

Акики Д. и др. Университет "Нотр-Дам", г. Триполи, Ливан. Снижение уровня боковых лепестков зеркальных антенн методом позиционирования металлических полосок в раскрыве. * \text{[}Электронный ресурс\text{]} url: \url{https://hi.booksc.eu/book/36545448/daff7f} (дата обращения \today) %* \text{[}Электронный ресурс\text{]} url: \url{https://pnd-st.ru/ot-chego-zavisit-uroven-bokovyh-lepestkov-sposoby-umensheniya/} (дата обращения \today)






\end{thebibliography}




